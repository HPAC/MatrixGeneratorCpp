\documentclass{article}
\usepackage[utf8]{inputenc}
\usepackage{graphicx}
\usepackage{url}
\usepackage{pbox}
\usepackage{float}
\usepackage{placeins}

% Begin Listings for C++
\usepackage{color}
\usepackage{float}
\usepackage{listings}
\lstloadlanguages{C++}

\lstset{
language=C++,                       % choose the language of the code
basicstyle=\footnotesize\ttfamily,  % the size of the fonts that are used for the
                                    % code
%numbers=left,                       % where to put the line-numbers
numberstyle=\tiny,                  % the size of the fonts that are used for the
                                    % line-numbers
stepnumber=1,                       % the step between two line-numbers. If it's
                                    % 1 each line will be numbered
numbersep=5pt,                      % how far the line-numbers are from the code
%backgroundcolor=\color{gray},      % choose the background color. You must add
                                    % \usepackage{color}
showspaces=false,                   % show spaces adding particular underscores
showstringspaces=false,             % underline spaces within strings
showtabs=false,                     % show tabs within strings adding particular
                                    % underscores
keywordstyle=\bfseries\color{darkblue},  % color of the keywords
commentstyle=\color{darkgreen},     % color of the comments
stringstyle=\color{darkred},        % color of strings
captionpos=b,                       % sets the caption-position to top
tabsize=2,                          % sets default tabsize to 2 spaces
frame=tb,                       % adds a frame around the code
breaklines=true,                    % sets automatic line breaking
breakatwhitespace=false,            % sets if automatic breaks should only happen
                                    % at whitespace
escapechar=\%,                      % toggles between regular LaTeX and listing
belowskip=0.3cm,                    % vspace after listing
morecomment=[s][\bfseries\color{darkblue}]{struct}{\ },
morecomment=[s][\bfseries\color{darkblue}]{class}{\ },
morecomment=[s][\bfseries\color{darkblue}]{public:}{\ },
morecomment=[s][\bfseries\color{darkblue}]{public}{\ },
morecomment=[s][\bfseries\color{darkblue}]{protected:}{\ },
morecomment=[s][\bfseries\color{darkblue}]{private:}{\ },
morecomment=[s][\bfseries\color{black}]{operator+}{\ },
xleftmargin=0.1cm,
%xrightmargin=0.1cm,
}
\newfloat{code}{H}{myc}
\definecolor{darkblue}{rgb}{0,0,.6}
\definecolor{darkred}{rgb}{.6,0,0}
\definecolor{darkgreen}{rgb}{0,.6,0}
\definecolor{red}{rgb}{.98,0,0}
\definecolor{gray}{rgb}{.6,.6,.6}
% End Listings for C++

\title{Linear algebra libraries in C++}

\begin{document}

\maketitle

\newcommand{\template}[2]{#1 $\langle$ #2 $\rangle$} 
\section{Basic information}

\subsection{MTL4}

\begin{table}[ht]
\centering
\resizebox{\textwidth}{!}{
\begin{tabular}{|c|c|}
  \hline
  Library type & Only template headers \\
  \hline
  Build type & No build, CMake config files \\
  \hline
  Namespace & mtl \\
  \hline
  Vector type & \template{dense\_vector}{T} \\
  \hline
  Matrix type & \template{dense2D}{T} or \template{matrix}{T} \\
  \hline
  Default ordering & row-major \\
  \hline
  Special types & \pbox{20cm}{banded and triangular views, but not suitable for HPC \\ compressed matrix (sparse) advised for triangular\footnote{ \url{http://www.simunova.com/docs/mtl4/html/banded__matrices.html} } } \\
  \hline
  Predefined generators & Only diagonal, Hessian and Laplacian\\
  \hline
\end{tabular}}
\end{table}

\subsection{Armadillo}

\begin{table}[ht]
\centering
\resizebox{\textwidth}{!}{
\begin{tabular}{|c|c|}
  \hline
  Library type & Template headers and shared library \\
  \hline
  Build type & CMake build + CMake config file \\
  \hline
  Namespace & arma \\
  \hline
  Vector type & \template{Row}{T} or \template{Col}{T} \\
  \hline
  Matrix type & \pbox{20cm}{\template{Mat}{T} or mat, fmat, cx\_mat, cx\_fmat \\ other templated types may not be allowed \\ for functions depending on LAPACK or ATLAS } \\
  \hline
  Default ordering & column-major \textbf{ONLY} \\
  \hline
  Special types & None mentioned in documentation \\
  \hline
  Predefined generators & \pbox{20cm}{Fill constructors (one, zero, eye, rand) \\ Generators: eye, linspace, one, rand, Toeplitz}\\
  \hline
\end{tabular}}
\end{table}

\subsection{Eigen}

\begin{table}[ht]
\centering
\resizebox{\textwidth}{!}{
\begin{tabular}{|c|c|}
  \hline
  Library type & Template\footnote{Lib removed} \\
  \hline
  Build type & CMake build + CMake config file \\
  \hline
  Namespace & Eigen \\
  \hline
  Vector type & VectorXf/d or RowVectorXf/d \\
  \hline
  Matrix type & MatriXf/d \\
  \hline
  Default ordering & column-major \\
  \hline
  Special types & \pbox{20cm}{DiagonalMatrix, triangular view through \\ triangularView(), selfadjointView()} \\
  \hline
  Predefined generators & Zero, Ones, Constant, Random \\
  \hline
\end{tabular}}
\end{table}

\FloatBarrier

\subsection{Blaze}

\begin{table}[ht]
\centering
\resizebox{\textwidth}{!}{
\begin{tabular}{|c|c|}
  \hline
  Library type & Template \\
  \hline
  Build type & None, configured Makefile for unused library \\
  \hline
  Namespace & blaze \\
  \hline
  Vector type & \template{DynamicVector}{T, (rowVector, columnVector)} \\
  \hline
  Matrix type & \template{DynamicMatrix}{T, (rowMajor, columnMajor)} \\
  \hline
  Default ordering & row-major\\
  \hline
  Special types & Adaptors for triangle and symmetric matrices \\
  \hline
  Predefined generators & None\\
  \hline
\end{tabular}}
\end{table}


\section{Matrix operations}

\begin{itemize}
\item \textbf{G}lobal - function in library namespace
\item \textbf{O}bject - method defined in object
\item \textbf{N}one 
\end{itemize}

\begin{itemize}
\item \textbf{I}nplace - modify current object
\item \textbf{R}eturn - new value or object
\end{itemize}

\begin{table}[ht]
\centering
\resizebox{\textwidth}{!}{
\begin{tabular}{|c|c|c|c|c|}
  \hline
  - & MTL4 & Armadillo & Eigen & Blaze \\
  \hline
  Norm & \textbf{GR} & \textbf{GR} & \textbf{OR} & \textbf{N} \\ 
  \hline
  Norms & 1, 2, Inf, Frobenius & 1, 2, Inf, Fro & Euclidian & \textbf{N} \\
  \hline
  Normalize & \textbf{N} & \textbf{GR} & \textbf{OI}, \textbf{OR} & \textbf{N} \\ 
  \hline
  Trace & \textbf{GR} & \textbf{GR} & \textbf{OR} & \textbf{N} \\
  \hline
  Transpose, conjugate & \textbf{GR} & \textbf{GR} & \textbf{OI}, \textbf{OR} & \textbf{OI}, \textbf{OR} \\
  \hline
  Determinant & \textbf{N} & \textbf{GR} & \textbf{OR} & \textbf{GR} \\
  \hline
  Inversion & \textbf{GR} & \textbf{GR}, \textbf{GI}, \textbf{OI} & \textbf{OR} & \textbf{GR},\textbf{OR} \\
  \hline
  Inversion algorithms & ? & Pseudoinverse on SVD & LU only & LU, Cholesky, Bunch-Kaufman \\
  \hline
\end{tabular}}
\end{table}

\subsection{Decompositions}

\begin{table}[ht]
\centering
\resizebox{\textwidth}{!}{
\begin{tabular}{|c|c|c|c|c|}
  \hline
  - & MTL4 & Armadillo & Eigen & Blaze \\
  \hline
  LU & lu & lu & \template{FullPivLU}{T} and PartialPivLU & lu \\
  \hline
  Cholesky & No & chol & \textbf{llt and ldlt} & llh \\
  \hline
  QR & No & qr & \textbf{householderQr} & qr \\
  \hline
  SVD & svd & svd & \textbf{jacobiSVD and BDCSVD} & No \\
  \hline
  Schur & No & schur & No & No \\
  \hline
  LE solver & lu\_solve & solve & Multiple solvers & LAPACK bindings\\
  \hline
\end{tabular}}
\end{table}

\FloatBarrier

\section{Datatypes}

\subsection{Special matrix types}

We consider the problem of defining storage for temporal results in a sequence of operations. This problem is unknown in dynamically typed languages such as MATLAB or Python but it becomes an issue strongly static environment of C++. 

\begin{code}[h]
\begin{lstlisting}[
    caption={Sequence of linear algebra operations.},
    label={lst:sequence}]
typedef /** **/ matrix_t;
matrix_t A = init(), B = init(), D = init();
matrix_t C = f(A, B);
matrix_t E = g(C, D);
\end{lstlisting}
\end{code}

The programmer is required to explicitly define matrix type at each stage of operation. \textit{matrix\_t} refers to a general dynamic matrix type defined by the library. If the user input in \textit{init} function includes matrix properties, such as symmetry, triangularity or tridiagonality, those properties may be preserved in operations defined in functions \textit{f} and \textit{g}. We find three questions here: how to create and define matrices with special types, does the library preserve and reuse the information about matrix to optimize computations and how can we define matrices \textit{C} and \textit{E} without preventing library from applying optimized algorithms?

Expression Templates and its compile-time type deduction implies separate types to store additional information on matrices. If an additional type is introduced, e.g. \textit{matrix\_symm\_t}, definition of matrix C becomes problematic. Type must be known at compilation time and it becomes the user responsibility to analyze function \textit{f} and decide whether \textit{matrix\_t}, \textit{matrix\_symm\_t} or any other type is suitable at that moment. If the user declares a special matrix type conflicting with results from \textit{f}, e.g. generic matrix is assigned to a symmetric one, program may fail to compile due to lack of assignment operator, it may introduce unnecessary overhead caused by a runtime test to verify if assignment operation is valid or in libraries without such facility the results may be incorrect. A safer and less involving for the user options is to use generic matrix everywhere. However when a different type is returned from \textit{f} will have two consequences: potential optimizations for symmetric matrices in \textit{g} are not applied and assignment to generic matrix from a special type may result in an unnecessary call to copy constructor.

C++11 introduced keyword \textit{auto} for automatic type deduction. Direct application of \textit{auto} in Expression Templates is going to deduce node type representing computation. It may defer computations due to lazy evaluation policy and it is generally not advised to use auto with ET libraries\cite{eigenAuto}. It seems that this issue may be resolved with enforced evaluation.

\begin{code}[h]
\begin{lstlisting}[
    caption={Auto keyword.},
    label={lst:sequence}]
typedef /** **/ matrix_t;
matrix_t A = init(), B = init(), D = init();
// May be unsafe, impossible to benchmark
auto C = f(A, B);
// Should be safe
auto D = evaluate(f(A,B));
\end{lstlisting}
\end{code}

\begin{itemize}
\item how to create matrices with special properties?
\item is the library capable of using this information at different stages of computation? 
\item how to \textit{dynamically} define matrices?
\end{itemize}

\subsubsection{Blaze}
Information valid at December 2016.
\begin{itemize}
\item adaptors to create symmetric, Hermitian and triangular matrices\cite{blazeAdaptors}
\item yes\cite{blazeEval}
\item we have requested this feature and it is quite straightforward due to result type defined in each ET node\cite{blazeEval}
\end{itemize}

Special matrix types:
\begin{itemize}
\item \textbf{SymmetricMatrix} \\
Symmetry is enforced when entries are modified. It doesn't matter which part of matrix is filled.
\item \textbf{HermitianMatrix} \\
Hermitian symmetry is enforced when entries are modified. It doesn't matter which part of matrix is filled.
\item \textbf{TriangularMatrix} \\
For \textbf{Lower} and \textbf{Upper}: XMatrix, UniXMatrix with diagonal elements equal to one, StrictlyXMatrix with zeros on diagonal and a DiagonalMatrix 
\end{itemize}

\subsubsection{Eigen}
Information valid at December 2016. \\
\begin{itemize}
\item views for symmetric, Hermitian and triangular matrices\cite{eigenViewSymm}\cite{eigenViewTriang}; no support for band matrices\cite{eigenBand}
\item rather unlikely\cite{eigenEval}
\item we have not been given a clear answer\cite{eigenEval}, documentation on auto keyword suggests that evaluation \textit{expression}.eval() should be safe
\end{itemize}

The main drawback of their model is that views are not standalone objects. Not only it makes impossible to save information about triangularity or symmetry in a new matrix but it makes much harder to implement a generator of matrices. A view should be used in computations but the underlying matrix has to be kept alive. Possible solutions: a generator object storing references to all special matrices and returning only views, a wrapper storing both matrix and view and implementing the necessary matrix interface, inheriting from Eigen matrix type.

The interface of matrix views is rather limited and may not be sufficient for some applications.
\begin{itemize}
\item \textbf{SelfAdjointView} \\
May refer to either Lower or Upper part of matrix; 'opposite triangular part is never referenced and can be used to store other information'
\item \textbf{TriangularView} \\
For \textbf{Lower} and \textbf{Upper} additionally: UnitX and StrictlyX. 
\end{itemize}

\subsubsection{MTL4}
Information valid at December 2016. \\
\begin{itemize}
\item no support for symmetry, views for banded and triangular matrices\cite{mtlView} but they are not advised: 'this is not recommended in high-performance applications because the lower triangle is still traversed while returning zero values'; 
\item very unlikely
\item auto is possible to use\cite{mtlAuto} with mtl::traits::evaluate; latter feature is not properly documented hence we may not rely on it completely
\end{itemize}

There seems to be no native support for defining matrices with specific properties.

\subsection{Imported datatypes}

\subsubsection{Blaze}

Imported matrix is blaze::CustomMatrix\cite{blazeCustom} \\
There are two important template parameters for custom matrices: \textbf{alignment} and \textbf{padding} for vectorized instructions. Either memory is allocated with dedicated Blaze allocator (or any other allocator aligning memory) or memory allocation process is shifted to matrix. Alignment may have an impact on performance, padding definitely has a significant positive impact according to Blaze.\\
However, padding makes initialization much harder. It may be easier to just perform copy from an unpadded memory storage. Thus one can reuse initialization procedure with other libraries.

\subsubsection{Eigen}
Imported matrix is eigen::Map\cite{eigenImport} \\
Array may be aligned or unaligned.

\subsubsection{Armadillo}
One can initialize a matrix from an array\cite{armaImport}, with or without an additional allocation. No information about alignment or padding.

\subsubsection{MTL4}
No clear information in documentation, only that a matrix may be initialized from a regular array\cite{mtlImport}. No information about alignment or padding.

\subsection{Expression Templates}

\subsubsection{Forced evaluation}
\begin{itemize}
\item \textbf{MTL4} mtl::traits::evaluate \\
\item \textbf{Eigen} \textit{expression}.eval() \\
\item \textbf{Armadillo} \textit{expression}.eval() \\
\item \textbf{Blaze} there is a function which does not work, created issue \\
\end{itemize}

\subsubsection{Aliasing}
\begin{itemize}
\item \textbf{MTL4} no information in docs \\
\item \textbf{Eigen} no checking, only default in GEMM, introduces a temporary. result matrix should be called with .noalias() to disable this feature\\ 
\item \textbf{Armadillo} implements aliasing check, no information about  \\
\item \textbf{Blaze} no clear information, it may not be a problem \\
\end{itemize}


\begin{thebibliography}{9}

\bibitem{blazeAdaptors}
\url{https://bitbucket.org/blaze-lib/blaze/wiki/Adaptors}

\bibitem{blazeEval}
\url{https://bitbucket.org/blaze-lib/blaze/issues/72/functionality-to-enforce-non-lazy}

\bibitem{blazeCustom}
\url{https://bitbucket.org/blaze-lib/blaze/wiki/Matrix%20Types#!custommatrix}

\bibitem{eigenViewSymm}
\url{https://eigen.tuxfamily.org/dox/classEigen_1_1SelfAdjointView.html}

\bibitem{eigenViewTriang}
\url{https://eigen.tuxfamily.org/dox/classEigen_1_1TriangularView.html}

\bibitem{eigenAuto}
\url{https://eigen.tuxfamily.org/dox/TopicPitfalls.html}

\bibitem{eigenBand}
\url{https://forum.kde.org/viewtopic.php?f=74&t=130666}

\bibitem{eigenEval}
\url{https://forum.kde.org/viewtopic.php?f=74&t=137795}

\bibitem{eigenImport}
\url{https://eigen.tuxfamily.org/dox/group__TutorialMapClass.html}

\bibitem{mtlAuto}
\url{http://www.simunova.com/docs/mtl4/html/cppeleven__intro.html}

\bibitem{mtlView}
\url{http://www.simunova.com/docs/mtl4/html/banded__matrices.html}

\bibitem{mtlImport}
\url{http://www.simunova.com/docs/mtl4/html/matrix__insertion.html}

\bibitem{armaImport}
\url{http://arma.sourceforge.net/docs.html#Mat}

\end{thebibliography}



\end{document}
